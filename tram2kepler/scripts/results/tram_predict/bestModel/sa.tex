\begin{tabular}{rll}
\toprule
Unnamed: 0 & segment & label(s) \\
\midrule
1 & In this campaign, the suspected Russian threat actors use several highly obfuscated and & \{'T1027'\} \\
3 & highly obfuscated and under-development custom loaders to infect those involved in the cryptocurrency & \{'T1027'\} \\
5 & distributed to victims via phishing attempts or through social explains how to amend & \{'T1566.001'\} \\
7 & in its operation. The  & \{'T1055'\} \\
9 & Jobs   Trend Micro (NO) MD5 1693D0A858B8FF3B83852C185880E459 SHA-1 5F1536F573D9BFEF21A4E15273B5A9852D3D81F1 SHA- 03B9D7296B01E8F3FB3D12C4D80FE8A1BB0AB2FD76F33C5CE11B40729B75FB23 256 File & \{'T1055'\} \\
11 & 03B9D7296B01E8F3FB3D12C4D80FE8A1BB0AB2FD76F33C5CE11B40729B75FB23 256 File 367.00 KB (375808 bytes) size The initial stage of Enigma, & \{'T1055'\} \\
13 & C  . Its primary objective is to download, deobfuscate, decompress, and launch the secondary stage payload. The malware incorporates & \{'T1140'\} \\
15 & hashing, string encryption, and irrelevant code. Before delving into the analysis of "EnigmaDownloader\_s001," & \{'T1027'\} \\
16 & code. Before delving into the analysis of "EnigmaDownloader\_s001," let's  & \{'T1055'\} \\
18 & examine how the malware decrypts strings and resolves hashed Windows APIs. By understanding & \{'T1140'\} \\
20 & to enhance code legibility, we have substituted all hashes with the corresponding function & \{'T1027'\} \\
22 & API Hashing: API hashing is a technique employed by malware to conceal the & \{'T1027'\} \\
24 & of potentially suspicious APIs (functions) from static detection. This technique helps the malware disguise its activities and evade & \{'T1027'\} \\
26 & technique is to make the process of understanding the code more time-consuming and & \{'T1055'\} \\
28 & the following custom MurmurHash:   This website uses cookFieigs uforer 7w.e Cbussitteo mfu & \{'T1055'\} \\
30 & cof murmur hash analytics, personalization, social media functionality and advertising. Our Cookie Notice & \{'T1027'\} \\
32 & of the APIs needed, then importing them dynamically at runtime. The Windows API o & \{'T1106'\} \\
34 & functions such as LoadLibrary and others. The use of standard methods like GetProcAddress & \{'T1106'\} \\
36 & since the malware author changed some of the constant values in the hash & \{'T1027'\} \\
38 & personalization, social media functionality and advertising. Our Cookie Notice provides more information and & \{'T1027'\} \\
40 & kernel32\_BasepSetFileEncryptionCompression 0xA3FE987 : advapi32\_RegDeleteKeyW 0x1CA6703 : advapi32\_RegCreateKeyA 0x24EBD39 : kernel32\_lstrlenA 0x69F38C6 : kernel32\_RegSetValueExA & \{'T1106'\} \\
42 & passes two arguments to the "mw\_resolveAPI" function. The  & \{'T1059.003'\} \\
44 & index number (in this case 0xA   Kernel32.dll), while the second argument is & \{'T1055'\} \\
46 & it, and decrypts the corresponding library name value as shown in the bottom & \{'T1140'\} \\
48 & is the list of decrypted library names:   This website uses cookies for & \{'T1140'\} \\
50 & address of an exported function. The techniques used to decrypt strings and resolve API hashes in both the & \{'T1140'\} \\
52 & in both the stage 1 and stage 2 payloads are identical. Figure 11. & \{'T1055'\} \\
54 & of this process, we can then proceed with our analysis.   Upon execution, & \{'T1055'\} \\
56 & as shown in Figure 14. Figure 14. Constructing  & \{'T1055'\} \\
58 & Figure 18. The code responsible for decrypting the next stage payload  & \{'T1140'\} \\
60 & If the  & \{'T1140'\} \\
62 & decompress the payload, the malware uses Microsoft Cabinet's explains how to amend your & \{'T1140'\} \\
64 & demonstrates how the malware downloads, deobfuscates, and decompresses  & \{'T1140'\} \\
66 &   Trend Micro (NO) Figure 21. Payload deobfuscation and decompression Before executing the payload, the malware attempts to & \{'T1140'\} \\
68 & (509440 bytes) File size The second stage payload, UpdatTask.dll, is a dynamic-link library & \{'T1027'\} \\
70 & in C   that comprises two export functions (DllEntryPoint and Entry). The malicious code & \{'T1055'\} \\
72 & simply as a regular user using the GetTokenInformation API. If the malware fails to obtain elevated privileges, it & \{'T1106'\} \\
74 & the disablement of Windows Defender and proceed to download and execute the next & \{'T1562.001'\} \\
76 & Trend Micro (NO) Figure 25. Checking the process privileges If the process successfully obtains elevated privileges, it proceeds & \{'T1055'\} \\
78 &   Trend Micro (NO) SHA-256 4429f32db1cc70567919d7d47b844a91cf1329a6cd116f582305f3b7b60cd60b Name Driver.SYS Description Malicious drivers reduce the & \{'T1027'\} \\
80 & by exploiting the iqvw64e.sys Intel driver. Testing on this has reportedly been conducted & \{'T1027'\} \\
82 & by creating scheduled tasks.   This website uses cookies for website functionality, tra & \{'T1053.005'\} \\
84 & Figure 28. Malware persistence is achieved via scheduled tasks (click the image for a larger version) Finally, the & \{'T1053.005'\} \\
86 & used in the  & \{'T1055'\} \\
88 & without terminating the process.   Figure 30. Microsoft defender token integrity modi & \{'T1055'\} \\
90 & The code snippets in Figure 31 demonstrate how the malware performs explains how & \{'T1055'\} \\
92 & to retrieve the command. Upon receiving the runassembly command, the malware downloads the & \{'T1059.003'\} \\
94 & and 4D2FB518C9E23C5C70E70095BA3B63580CAFC4B03F7E6DCE2931C54895F13B2C 2e5xp6lains how to amend your cookie settings. Learn more Business   & \{'T1055'\} \\
96 &   "0";   public static string Autorun   "1"; public static string StartDelay   "0"; public static string & \{'T1027'\} \\
98 & Trend Micro (NO) public static string KeyloggerModule   "0"; public static string ClipperModule   "0"; public static string & \{'T1027'\} \\
100 & services URLs, are encrypted with the AES algorithm in cipher-block chaining (CBC) mode. & \{'T1027'\} \\
102 & website uses cookies for website functionality, tra & \{'T1027'\} \\
104 & this case highlights the evolving nature of modular malware that employ highly obfuscated and evasive techniques along with the utilization of continuous integration & \{'T1027'\} \\
106 & or phishing attempts that o & \{'T1566.001'\} \\
108 &   Trend Micro (NO) CONTACT US SUBSCRIBE Related Articles Trend Micro Collaborated with & \{'T1027'\} \\
110 & Micro Collaborated with Interpol in Cracking Down Grandoreiro Banking Trojan NCSC Says Newer & \{'T1027'\} \\
112 & personalizatRioens, soouciracl mesedia functionality and advertising. Our Cookie Notice provides more information and & \{'T1027'\} \\
\bottomrule
\end{tabular}
